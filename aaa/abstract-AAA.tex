

Inductive-inductive types \cite{nordvallinductive, gabephd} (IITs) allow the
mutual definition of a type and, for example, a type family indexed over that
type.  This can be used to encode collections of types which are intricately
inter-related such as the syntax of type theory itself \cite{ttintt}, where we
define a type $\Con : \UU$ of contexts at the same time as a type of types over
a context: $\Ty : \Con \to \UU$. We show that IITs are reducible to indexed
inductive types in a setting with uniqueness of identity proofs and function
extensionality.

\subsubsection*{Signatures and initial algebras for IITs}

In \cite{qiit}, a syntax for signatures of quotient inductive-inductive types
(QIITs) is given. It is given as domain-specific type theory, where every typing
context can be viewed as a listing of type, value and equality constructors. It
is also shown that initial algebras for all QIITs can be constructed from terms
of the syntax of signatures.

We can get inductive-inductive signatures by taking the equality-free fragment
of QIIT signatures. We also restrict signatures so that they are closed (cannot
refer to types external to a signature). In this case it also holds that all
IITs are constructible from terms of the syntax of II signatures.

However, this syntax of II signatures is given as a QIIT, and thus at this stage
we can only show that if we assume the existence of this particular QIIT, then
all IITs exist. We strengthen this result by giving a construction of the syntax
of II signatures from indexed inductive types, thereby showing that all IITs are
in fact constructible from indexed inductive types.

\subsubsection*{Constructing the syntax of II signatures}

This can be viewed as an instance of the initiality problem popularized by
Voevodsky: we have an intrinsically typed categorical notion of model for a
particular type theory (the theory of II signatures), and we aim to construct
the initial model, using inductively defined preterms and well-formedness
relations.

However, the initiality proof in our case is simpler than in general, because
the theory of II signatures does not contain $\beta$-rules, and thus it is
possible to construct the initial model using only $\beta$-normal preterms,
avoiding the usage of quotients.

Hence, we first define normal preterms and typing relations on them, using indexed
inductive types. Then, we use well-typed preterms to construct a model of the theory
of II signatures. Lastly, we show that the constructed model is initial. We explain
the last step in more detail.

\subsubsection*{Initiality of the term model}

Initiality means that we have a \emph{recursion principle} for II signatures,
and recursors are also unique. This is less convenient in practice than
\emph{induction}, but \cite{qiit} shows unique recursion to be equivalent to
induction, and in our case it is easier to consider the former.

To show initiality, we consider an arbitrary model for the theory of II
signatures, and exhibit a unique morphism from the previously given term model
to it. We do this in the following steps. Each step is preformed by induction on
the presyntax.

\begin{enumerate}
  \item We define a relation between the presyntax and the given model.
  \item We show that the relation is right-unique.
  \item We show that the relation is preserved by substitution.
  \item We show that the relation is left-total on well-typed presyntax.
  \item Since now the relation is shown to be functional, we can use it to
        build a model morphism.
  \item We show that the model morphism is unique.
\end{enumerate}

Streicher's previous construction \cite{streicher2012semantics} used a family of
partial functions from the presyntax to a given model, which functions are later
shown to be total on well-formed presyntax. In contrast, we use a
\emph{relation} which is later shown to be functional. For our use case, the
relational approach seemed to be more convenient in a mechanized setting, and it
could be worthwhile to try it in initiality proofs for other type theories as
well.

\subsubsection*{Formalization}
We formalized in Agda
(\url{https://github.com/amblafont/UniversalII/blob/cwf-syntax/Cwf/}) the
construction of the syntax of IIT signatures. Separately, there is an Agda
formalization for \cite{qiit} in
\url{https://bitbucket.org/akaposi/finitaryqiit}, which includes the
construction of IITs from the syntax of II signatures.

Note that the \cite{qiit} formalization assumed definitional computation rules
for induction on signatures, while our current construction of signatures only
provides propositional computation rules. However, we already assume uniqueness
of identity proofs (UIP) along with function extensionality, and also use limited
equality reflection in the form of Agda rewrite rules. Hence, our formalization
is largely in an extensional setting, and thus the propositional-definitional
mismatch is not essential. Repeating these constructions without UIP and equality
reflection is a potential line of future work. Another future research would be to
extend the current result to open and infinitary signatures and QIITs.
